% 17.01.2006, c
% 22.01.2006
% 30.03.2007, from exhaust_noise_seminar_slides.tex
% 03.04.2007
\documentclass[10pt,t]{beamer}

\usepackage{listings}
\usepackage{bm}
\usepackage{amsmath}

\def\figDir{figures}
\newcommand{\red}[1]{{\color{red}{#1}}}
\newcommand{\blue}[1]{{\color{blue}{#1}}}
\definecolor{lblue}{rgb}{0.0,0.4,1.0}
\newcommand{\lblue}[1]{{\color{lblue}{#1}}}
\definecolor{dgreen}{rgb}{0.0,0.4,0.11}
\newcommand{\dgreen}[1]{{\color{dgreen}{#1}}}

\def \pder#1#2{\frac{\partial #1}{\partial #2}}
\newcommand{\dvg}{\mathop{\rm div}}
\newcommand{\ul}[1]{\underline{#1}}

%%
% Beamer related packages & defs.
\usepackage{multimedia}

\mode<handout>{\beamertemplatesolidbackgroundcolor{black!5}}
\mode<presentation>
{
%  \usetheme{Frankfurt}
  \useoutertheme[subsection=false]{smoothbars}
  \setbeamerfont{block title}{size={}}
%  \useinnertheme{rectangles}
  \useinnertheme[shadow=true]{rounded}
  \useinnertheme{rounded}
%  \useinnertheme{circles}
%  \useinnertheme{inmargin}
  \setbeamercovered{transparent}
  \usecolortheme{seahorse}
  \usecolortheme{rose}
  \beamertemplategridbackground[0.2cm]
  \beamertemplatetransparentcovereddynamic

  \setbeamertemplate{navigation symbols}{
    \insertbackfindforwardnavigationsymbol
    \hspace{1mm} \hbox{\normalsize \insertframenumber/\inserttotalframenumber}
  }
}

%%
% Title page info.
\title{Python + FEM}
\subtitle{Introduction to SFE}

\author{Robert Cimrman}

\institute[University of West Bohemia]
{
  Department of Mechanics \& New Technologies Research Centre \\
  University of West Bohemia \\
  Plze\v{n}, Czech Republic
}

\titlegraphic{
  \includegraphics[height=1.5cm,viewport=96 122 504 527,clip]{\figDir/zcu}
}

\date{April 3, 2007, Plze\v{n}}

\subject{Software design}

% \AtBeginSection[]
% {
%   \begin{frame}<beamer>
%     \frametitle{Outline}
%     \tableofcontents[currentsection]
%   \end{frame}
% }

\begin{document}

\frame{\titlepage}

\begin{frame}
  \frametitle{Outline}
  \tableofcontents
\end{frame}

\section{Introduction}


\subsection{Programming Languages}

\begin{frame}
  \frametitle{Programming Languages}
  \vspace*{-3mm}
  \begin{enumerate}
  \item<1-> compiled (fortran, C, C++, Java, \dots)
    \vspace*{-5mm}
    \begin{columns}[t]
      \begin{column}{0.52\linewidth}
        \begin{exampleblock}<2->{Pros}
          \begin{itemize}
          \item speed, speed, speed, \dots, did I say \blue{speed}?
          \item large code base (legacy codes), tradition
          \end{itemize}
        \end{exampleblock}
      \end{column}
      \begin{column}{0.55\linewidth}
        \begin{alertblock}<3->{Cons}
          \begin{itemize}
          \item (often) complicated build process, recompile after any change
          \item low-level $\Rightarrow$ lots of lines to get basic stuff done
          \item code size $\Rightarrow$ maintenance problems
          \item \red{static!}
          \end{itemize}
        \end{alertblock}
      \end{column}
    \end{columns}
  \item<4-> interpreted or scripting (sh, tcl, matlab, perl, ruby, python,
    \dots) \vspace*{-5mm}
    \begin{columns}[t]
      \begin{column}{0.52\linewidth}
        \begin{exampleblock}<5->{Pros}
          \begin{itemize}
          \item no compiling
          \item (very) high-level $\Rightarrow$ a few of lines to get (complex)
            stuff done
          \item code size $\Rightarrow$ easy maintenance
          \item \blue{dynamic!}
          \item (often) large code base
          \end{itemize}
        \end{exampleblock}
      \end{column}
      \begin{column}{0.55\linewidth}
        \begin{alertblock}<6->{Cons}
          \begin{itemize}
          \item many are relatively new
          \item lack of speed, or even utterly \red{slow}!
          \end{itemize}
        \end{alertblock}
      \end{column}
    \end{columns}
  \end{enumerate}
\end{frame}

\begin{frame}
  \frametitle{Example Python Code}

  \includegraphics[height=0.9\textheight]{\figDir/examples}
\end{frame}


\lstset{language=C}
\defverbatim{\verbexamples}{
\small
\begin{lstlisting}
char s[] = "sel tudy, mel dudy. a sel opravdu tudy.";

int strfind(char *str, char *tgt)
{
	int tlen = strlen(tgt);
	int max = strlen(str) - tlen;
	register int i;

	for (i = 0; i < max; i++) {
	    if (strncmp(&str[i], tgt, tlen) == 0)
		return i;
	}
	return -1;
}

int i1;

i1 = strfind( s, "tudy" );
printf( "%d\n", i1 );
\end{lstlisting}
}
\lstset{language=}

\begin{frame}
  \frametitle{Example C Code}

  \verbexamples
\end{frame}



\subsection{Programming Techniques}

\begin{frame}
  \frametitle{Programming Techniques}
  Random remarks:
  \begin{itemize}
  \item spaghetti code
  \item modular programming
  \item OOP
  \item \dgreen{design patterns} (gang-of-four, GO4)
  \item post-OOP --- dynamic languages
  \end{itemize}
  \begin{center}
    \uncover<2->{\dots all used, but I try to move down the list! :-)}
  \end{center}
\end{frame}

\section{Our choice}

\subsection{Mixing Languages --- Best of Both Worlds}

\begin{frame}
\frametitle{Mixing Languages --- Best of Both Worlds}  
\begin{alertblock}<1->{Low level: C}
  \
  \begin{itemize}
  \item FE matrix evaluations
  \item costly mesh-related functions (listing faces, edges, surface
    generation, \dots)
  \item \dots
  \item can use also \red{fortran}
  \end{itemize}
\end{alertblock}
\uncover<2->{\centerline{\LARGE +}}
\begin{exampleblock}<2->{High level: Python}
  \centerline{\href{http://www.python.org}{\blue{www.python.org}}}
  \centerline{``batteries included''}
  \begin{minipage}{0.5\linewidth}
    \begin{itemize}
    \item logic of the code
    \item \lblue{configuration} files!
    \item problem \blue{input} files!
    \end{itemize}
  \end{minipage}
  \hfill
  \begin{minipage}{0.4\linewidth}
    \includegraphics[width=\linewidth]{\figDir/python-logo}
  \end{minipage}
\end{exampleblock}
\end{frame}

\subsection{Python}

\begin{frame}
  \frametitle{Python I}
  \begin{quotation}
    Python\textregistered is a dynamic object-oriented programming language
    that can be used for many kinds of software development. It offers strong
    support for integration with other languages and tools, comes with
    extensive standard libraries, and can be learned in a few days. Many Python
    programmers report substantial productivity gains and feel the language
    encourages the development of higher quality, more maintainable code.
  \end{quotation}
  \vspace*{10mm}
  \centerline{\dots}
  \vspace*{10mm}
  \centerline{\blue{``batteries included''}}
\end{frame}

\begin{frame}
  \frametitle{Python II}
  1.1.16   Why is it called Python?
  \begin{quotation}
    At the same time he began implementing Python, Guido van Rossum was also
    reading the published scripts from "Monty Python's Flying Circus" (a BBC
    comedy series from the seventies, in the unlikely case you didn't know). It
    occurred to him that he needed a name that was short, unique, and slightly
    mysterious, so he decided to call the language Python.
  \end{quotation}

  1.1.17   Do I have to like "Monty Python's Flying Circus"?
  \begin{quotation}
    No, but it helps. :)
  \end{quotation}

  \hfill \dots General Python FAQ
  
\end{frame}


\begin{frame}
  \frametitle{Python III}

  NASA uses Python\dots
  \begin{center}
    \includegraphics[width=0.5\linewidth]{\figDir/nasa}
  \end{center}
  \dots so does Rackspace, Industrial Light and Magic, AstraZeneca, Honeywell,
  and many others.
  \vspace*{10mm}
  \begin{center}
    \uncover<2->{\href{http://wiki.python.org/moin/NumericAndScientific}
      {http://wiki.python.org/moin/NumericAndScientific}}
  \end{center}
\end{frame}




\subsection{Software Dependencies}

\begin{frame}
  \frametitle{Software Dependencies}
  \begin{enumerate}
  \item<1-> libraries \& modules
    \begin{dinglist}{226}
    \item<2-> \blue{SciPy}: free (BSD license) collection of numerical computing
      libraries for Python
      \begin{itemize}
      \item enables Matlab-like array/matrix manipulations and indexing
      \end{itemize}
    \item<3-> \lblue{UMFPACK}: very fast direct solver for sparse systems
      \begin{itemize}
      \item Python wrappers are part of SciPy :-)
      \end{itemize}
    \item<4-> output files (results) and data files in \blue{HDF5}:
      \begin{itemize}
      \item general purpose library and file format for storing scientific
        data, efficient storage and I/O
      \item \dots via \lblue{PyTables}
      \end{itemize}
    \item<5-> misc: \blue{Pyparsing}, \lblue{Matplotlib}
    \end{dinglist}
  \item<5-> tools
    \begin{dinglist}{226}
    \item<6-> \lblue{SWIG}: automatic generation of Python-C interface code
    \item<7-> \blue{Medit}: simple mesh and data viewer
    \item<7-> \lblue{ParaView}: VTK-based advanced mesh and data viewer in
      Tcl/Tk
    \item<7-> \blue{MayaVi/MayaVi2}: VTK-based advanced mesh and data viewer in
      Python
    \end{dinglist}
  \end{enumerate}
\end{frame}

\section{Example Problem}

\begin{frame}
  \frametitle{Shape Optimization in Incompressible Flow Problems}
  \vspace*{-4mm}
  \begin{block}<1->{Objective Function}
    \ 
    \uncover<1->{
      \begin{equation} \label{eq-ns-4}
        \Psi_{\Gamma}(u) \equiv \frac{\nu}{2}
        \int_{\Omega_c} |\nabla u|^2 \longrightarrow \min
      \end{equation}
    }
    \begin{itemize}
    \item<1-> minimize gradients of solution (e.g. losses) in $\Omega_c
      \subset \Omega$
    \item<2-> by moving \dgreen{design boundary} $\Gamma \subset \partial
      \Omega$
    \item<3-> perturbation of $\Gamma$ by vector field $\mathcal{V}$
      \begin{equation}
        \Omega(t) = \Omega + \{t\mathcal{V}(x)\}_{x \in \Omega} \quad \mbox{
          where } \mathcal{V} = 0 \mbox{ in } \bar\Omega_c \cup \partial \Omega
        \setminus \Gamma
      \end{equation}
    \end{itemize}
  \end{block}
  \vspace*{-3mm}
  \begin{center}
    \begin{minipage}{0.48\linewidth}
      \includegraphics[width=0.35\linewidth,angle=-90]
      {\figDir/fig-design-domain-flow.pdf}
    \end{minipage}
    \hfill
    \begin{minipage}{0.48\linewidth}
      \includegraphics[width=0.8\linewidth]{\figDir/geometry_kroucenak_c23}
    \end{minipage}
  \end{center}
\end{frame}


\begin{frame}
  \frametitle{Appetizer}
  \vspace*{-3mm}
  \begin{itemize}
  \item flow and spline boxes, left: initial, right: final \\
    \noindent
    \hspace*{-5mm}
      \begin{minipage}{0.48\linewidth}
        \includegraphics[width=\linewidth]{\figDir/kroucenak_c2dup_ini}
      \end{minipage}
      $\rightarrow$
      \begin{minipage}{0.48\linewidth}
        \includegraphics[width=\linewidth]{\figDir/kroucenak_c2dup_opt}
      \end{minipage}
    \item connectivity of spline boxes (6 boxes in different colours)
    \begin{center}
      \begin{minipage}{0.4\linewidth}
        \includegraphics[width=\linewidth]{\figDir/kroucenak_c2dup_cpconn}
      \end{minipage}
    \end{center}
  \end{itemize}
\end{frame}


\subsection{Continuous Formulation}

\begin{frame}
  \frametitle{Continuous Formulation}
  \begin{itemize}
  \item<1-> laminar steady state incompressible \blue{Navier-Stokes equations}
    \begin{eqnarray}
      -\nu \nabla^2 \ul{u} + \ul{u} \cdot \nabla \ul{u} + \nabla p & = & 0 \quad \mbox{ in }
      \Omega \label{eq-ns-1}\\
      \nabla \cdot \ul{u} & = & 0 \quad \mbox{ in } \Omega \label{eq-ns-2}
    \end{eqnarray}
    boundary conditions
    \begin{equation} \label{eq-ns-3}
      \begin{split}
        \ul{u} & = \ul{u}_0 \quad  \mbox{ on } \Gamma_{\rm in} \\
        -p \ul{n} + \nu \pder{\ul{u}}{\ul{n}} & = -\bar p \ul{n}
        \quad \mbox{ on } \Gamma_{\rm out} \\
        \ul{u} & = 0 \quad \mbox{ on } \Gamma_{\rm walls}
      \end{split}
    \end{equation}
  \item<2->
    \lblue{adjoint equations} for computing the adjoint solution $\ul{w}$:
    \begin{equation}\label{eq-ns-33}
      \begin{split}
        -\nu \nabla^2 \ul{w} + \nabla \ul{u} \cdot \ul{w} - \ul{u} \cdot \nabla
        \ul{w} - \nabla r & =
        \chi_{\Omega_c} \,\nu \nabla^2 \ul{u} \ \mbox{ in } \Omega \\
        \nabla \cdot \ul{w} & = 0 \quad \mbox{ in } \Omega
      \end{split}
    \end{equation}
    $\ul{w} = 0$ on $\Gamma_{\rm in} \cup \Gamma_{\rm walls}$ and
    $\chi_{\Omega_c}$ is the characteristic function of $\Omega_c$
  \end{itemize}
\end{frame}
\subsection{Weak Formulation}

\begin{frame}
  \frametitle{Weak Formulation}
  \begin{itemize}
  \item<1-> \blue{state problem}: find \ul{u}, p (from suitable function
    spaces) such that
    \begin{equation}\label{eq-ns-34a}
      \begin{split}
        &\int_\Omega \nu \nabla \ul{v} : \nabla \ul{u} + \int_\Omega (\ul{u}
        \cdot \nabla \ul{u})\cdot \ul{v} - \int_\Omega p \nabla \cdot \ul{v} =
        - \int_{\Gamma_{\rm out}} \bar{p} \ul{v} \cdot \ul{n} \quad \forall
        \ul{v} \in V_0 \;,
        \\
        & \int_{\Omega} q \nabla \cdot \ul{u} = 0 \quad \forall q \in Q
      \end{split}
    \end{equation}
  \item<2-> \lblue{adjoint equations}: find \ul{w}, r (from suitable function
    spaces) such that
    \begin{equation}\label{eq-ns-34b}
      \begin{split}
        & \int_\Omega \nu
          \nabla \ul{v} : \nabla \ul{w} + \int_\Omega (\ul{v} \cdot \nabla
          \ul{u})\cdot \ul{w} + \int_\Omega (\ul{u} \cdot \nabla
          \ul{v})\cdot \ul{w} + \int_\Omega r \nabla \cdot \ul{v} \\
        & = - \int_{\Omega_c} \nu \nabla \ul{v} : \nabla \ul{u} \quad \forall
        \ul{v} \in V_0 \;, \\
        & \int_{\Omega} q \nabla \cdot \ul{w} = 0\quad \forall q \in Q
      \end{split}
    \end{equation}
  \item<3-> + proper BC \dots
  \end{itemize}
\end{frame}

\subsection{Problem Description File}
%\lstset{language=Python}

%%%%%
\defverbatim\verbmesh{\scriptsize
\begin{lstlisting}
fileName_mesh = 'simple.mesh'
\end{lstlisting}
}
%%%%%
\defverbatim{\verbregions}{\scriptsize
\begin{lstlisting}
region_1000 = {
    'name' : 'Omega',
    'select' : 'elements of group 6',
}
region_0 = {
    'name' : 'Walls',
    'select' : 'nodes of surface -n r.Outlet',
}
region_1 = {
    'name' : 'Inlet',
    'select' : 'nodes by cinc( x, y, z, 0 )',
}
region_2 = {
    'name' : 'Outlet',
    'select' : 'nodes by cinc( x, y, z, 1 )',
}
region_100 = {
    'name' : 'Omega_C', # Control domain.
    'select' : 'nodes in (x > -25.9999e-3) & (x < -18.990e-3)',
}
region_101 = {
    'name' : 'Omega_D', # Design domain.
    'select' : 'all -e r.Omega_C',
}
\end{lstlisting}
}
%%%%%
\defverbatim{\verbebc}{\scriptsize
\begin{lstlisting}
ebc = {
    'Walls' : (('Walls', (3,4,5), 0.0 ),),
    'Inlet' : (('VelocityInlet_x', (4,), 1.0),
               ('VelocityInlet_yz', (3,5), 0.0)),
}
\end{lstlisting}
}
%%%%%
\defverbatim{\verbfields}{\scriptsize
\begin{lstlisting}
field_1 = {
    'name' : '3_velocity',
    'dim' : (3,1),
    'flags' : ('BQP',),
    'bases' : (('Omega', '3_4_P1B'),)
}

field_2 = {
    'name' : 'pressure',
    'dim' : (1,1),
    'flags' : (),
    'bases' : (('Omega', '3_4_P1'),)
}
\end{lstlisting}
}
%%%%%
\defverbatim{\verbvariables}{\scriptsize
\begin{lstlisting}
variables = {
    'u'   : ('field', 'parameter', '3_velocity', (3, 4, 5)),
    'w_0' : ('field', 'parameter', '3_velocity', (3, 4, 5)),
    'w'   : ('field', 'unknown', '3_velocity', (3, 4, 5), 0),
    'v'   : ('field', 'test', '3_velocity', (3, 4, 5), 'w'),
    'p'   : ('field', 'parameter', 'pressure', (9,)),
    'r'   : ('field', 'unknown', 'pressure', (9,), 1),
    'q'   : ('field', 'test', 'pressure', (9,), 'r'),
    'Nu'  : ('field', 'meshVelocity', 'pressure', (9,)),
}
\end{lstlisting}
}
%%%%%
\defverbatim{\verbequations}{\scriptsize
\begin{lstlisting}
equations = {
    'balance' :
    """+ dw_div_grad.Omega( fluid, v, w )
       + dw_convect.Omega( v, w )
       - dw_grad.Omega( v, r )
       = - dw_surface_ltr.Outlet( bpress, v )""",
    'incompressibility' :
    """dw_div.Omega( q, w ) = 0""",
}
\end{lstlisting}
}
%%%%%
\defverbatim{\verbequationsadjoint}{\scriptsize
\begin{lstlisting}
equations_adjoint = {
    'balance' :
    """+ dw_div_grad.Omega( fluid, v, w )
       + dw_adj_convect1.Omega( v, w, u )
       + dw_adj_convect2.Omega( v, w, u )
       + dw_grad.Omega( v, r )
       = - dw_adj_div_grad2.Omega_C( one, fluid, v, u )""",
    'incompressibility' :
    """dw_div.Omega( q, w ) = 0""",
}
\end{lstlisting}
}
%%%%%
\defverbatim{\verbmaterials}{\scriptsize
\begin{lstlisting}
material_2 = {
    'name' : 'fluid',
    'mode' : 'here',
    'region' : 'Omega',
    'viscosity' : 1.25e-1,
}
material_100 = {
    'name' : 'bpress',
    'mode' : 'here',
    'region' : 'Omega',
    'val'  : 0.0,
}
\end{lstlisting}
}

\begin{frame}
  \frametitle{Problem Description File I}
  \begin{itemize}
  \item<1-> a regular python file
    \red{$\Rightarrow$ all python machinery is accessible!}
  \item<2-> requires/recognizes several \blue{keywords} to define
    \begin{itemize}
    \item<2-> mesh
    \item<2-> regions $\rightarrow$ subdomains, boundaries, characteristic
      functions, \dots
    \item<2-> boundary conditions $\rightarrow$ Dirichlet
    \item<2-> fields $\rightarrow$ field approximation per subdomain
    \item<2-> variables $\rightarrow$ unknown fields, test fields, parameters
    \item<2-> equations
    \item<2-> materials $\rightarrow$ constitutive relations, e.g. newton
      fluid, elastic solid, \dots
    \item<2-> solver parameters $\rightarrow$ nonlinear solver, flow solver,
      optimization solver, \dots
    \item<2-> \dots
    \end{itemize}
  \end{itemize}
\end{frame}


\begin{frame}
  \frametitle{Problem Description File II}
  \vspace*{-5mm}
  \begin{itemize}
  \item<1-> \blue{mesh} \verbmesh
  \item<1-> \lblue{regions} \verbregions
  \end{itemize}
\end{frame}


\begin{frame}
  \frametitle{Problem Description File III}
  \begin{itemize}
  \item<1-> \blue{fields} \verbfields
  \item<2-> \lblue{variables} \verbvariables
  \end{itemize}
\end{frame}

\begin{frame}
  \frametitle{Problem Description File IV}
  \begin{itemize}
  \item<1-> \blue{Dirichlet BC} \verbebc
  \item<2-> \lblue{state equations} (compare with (\ref{eq-ns-34a}))
    \verbequations
  \end{itemize}
\end{frame}

\begin{frame}
  \frametitle{Problem Description File V}
  \begin{itemize}
  \item<1-> \blue{adjoint equations} (compare with (\ref{eq-ns-34b}))
    \verbequationsadjoint
  \item<2-> \lblue{materials} \verbmaterials
  \end{itemize}
\end{frame}

\section{Final slide :-)}

\subsection*{}

\begin{frame}
  \frametitle{Yes, the final slide!}
  \vspace*{-5mm}
  \begin{columns}[t]
    \begin{column}{0.52\linewidth}
      \begin{exampleblock}<1->{What is done}\
        \begin{itemize}
        \item basic FE element engine
          \begin{itemize}
          \item approximations up to P2 on simplexes (possibly with bubble)
          \item Q1 tensor-product approximation on rectangles
          \end{itemize}
        \item handling of fields and variables
          \begin{itemize}
          \item boundary conditions, DOF handling
          \end{itemize}
        \item FE assembling
        \item equations, terms, regions
        \item materials, material caches
        \item generic solvers (Newton iteration, direct linear system solution)
        \end{itemize}
      \end{exampleblock}
    \end{column}
    \begin{column}{0.55\linewidth}
      \begin{block}<2->{What is not done}
        \begin{itemize}
        \item general FE engine, with symbolic evaluation (\`a la SyFi - merge
          it?)
        \item documentation, unit tests
        \item organization of solvers
          \begin{itemize}
          \item fast problem-specific solvers
          \end{itemize}
        \item parallelization of both assembling and solving
          \begin{itemize}
          \item use PETSC?
          \end{itemize}
        \end{itemize}
      \end{block}
      \begin{alertblock}<3->{What will not be done (?)}
        \begin{itemize}
        \item GUI
        \item real symbolic parsing/evaluation of equations
        \end{itemize}
      \end{alertblock}
    \end{column}
  \end{columns}
\end{frame}

\section*{}

\begin{frame}
  \frametitle{This is not a slide!}
  \vspace*{-3mm}
  \includegraphics[width=\linewidth]{\figDir/foot_mesh}
\end{frame}

\end{document}
